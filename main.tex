% head
\documentclass[a4paper,12pt,DIV15]{scrartcl}

% meta data
\author{Martchus}
\title{Summary: \LaTeX}
\date{\today}
\subtitle{Most useful \LaTeX{} syntax and features summarized in one document}
\makeatletter
\let\thetitle\@title
\let\theauthor\@author
\title{\large{Summary} \\ ~ \\ \huge{\LaTeX} \\ ~}
\makeatother

% page geometry
\usepackage{geometry}
\geometry{a4paper,left=20mm,right=20mm,top=20mm,bottom=20mm}

% language and hyphenation
\usepackage[english]{babel}
\selectlanguage{\english}

% text encoding
\usepackage[T1]{fontenc}
\usepackage[utf8]{inputenc}

% placeholder text
\usepackage{lipsum}

% named references
\usepackage{nameref}

% colors
\usepackage{xcolor}

% math
\usepackage{amsmath}

% remove indention
\setlength{\parindent}{0pt}
\setlength{\parskip}{\baselineskip}

% line spacing
\usepackage{setspace}
\onehalfspacing

% \BibTeX
\usepackage{dtklogos}

% \minibox
\usepackage{minibox}

% \makecell
\usepackage{makecell}

% \ovalbox, \shadowbox, \doublebox, ...
\usepackage{fancybox}

% allows to modify appearance of headings (\sectionfont{}, ...)
\usepackage{sectsty}

% special characters
\usepackage{textcomp}

% euro symbol
\usepackage[official,right]{eurosym}

% commands for Pi fonts (Dingbats, Symbol, etc.)
\usepackage{pifont}

% comparsion operations
\usepackage{ifthen}

% space (after new commands)
\usepackage{xspace}

% bookmarks, hyperlinks, PDF settings
\usepackage{hyperref}
\hypersetup {
    pdftitle = {\thetitle},
    pdfauthor = {\theauthor},
    pdfkeywords = {LaTeX, summary, commands},
    colorlinks = {true},
    linkcolor = {blue},
    urlcolor = {blue}
}

% code formatting (lstlisting environment)
\usepackage{listings}

% backslash shortcut
\newcommand{\bs}{\textbackslash}

% reference shortcut
\newcommand{\fullref}[1]{\ref{#1}:~\nameref{#1}}

% no vertical spacing between paragraphs
\parskip 1ex

% table environments for command tables
\usepackage{tabularx}
\usepackage{array}

\newcolumntype{$}{>{\ttfamily \global\let\currentrowstyle\relax}}
\newcolumntype{^}{>{\currentrowstyle}}
\newcommand{\rowstyle}[1]{\gdef\currentrowstyle{#1}%
    #1\ignorespaces
}

\newenvironment{cmdtab}
{
    \tabularx{\textwidth}{$l ^X}
        \rowstyle{\normalfont\bfseries} Command & Effect \\
}{
    \endtabularx
}

\newenvironment{cmdtabx}[2]
{
    \tabularx{\textwidth}{$l ^X}
        \rowstyle{\normalfont\bfseries} #1 & #2 \\
}{
    \endtabularx
}

\newenvironment{cmdtabxx}[3]
{
    \tabularx{\textwidth}{$l $l $l}
        \rowstyle{\normalfont\bfseries} #1 & \rowstyle{\normalfont\bfseries} #2 & \rowstyle{\normalfont\bfseries} #3 \\
}{
    \endtabularx
}

\newenvironment{cmdtabxyy}[3]
{
    \tabularx{\textwidth}{$r ^l ^l}
        \rowstyle{\normalfont\bfseries} #1 & \rowstyle{\normalfont\bfseries} #2 & \rowstyle{\normalfont\bfseries} #3 \\
}{
    \endtabularx
}

\newenvironment{cmdtabxy}[3]
{
    \tabularx{\textwidth}{$l $l ^X}
        \rowstyle{\normalfont\bfseries} #1 & \rowstyle{\normalfont\bfseries} #2 & \rowstyle{\normalfont\bfseries} #3 \\
}{
    \endtabularx
}

\newenvironment{cmdtabxxx}[4]
{
    \tabularx{\textwidth}{$l $l $l $l}
        \rowstyle{\normalfont\bfseries} #1 & \rowstyle{\normalfont\bfseries} #2 & \rowstyle{\normalfont\bfseries} #3 & \rowstyle{\normalfont\bfseries} #4 \\
}{
    \endtabularx
}

\newenvironment{cmdtabxxxx}[2]
{
    \tabularx{\textwidth}{$l $l}
        \rowstyle{\normalfont\bfseries} #1 & \rowstyle{\normalfont\bfseries} #2 \\
}{
    \endtabularx
}

\newenvironment{cmdtabxxxxx}[5]
{
    \tabularx{\textwidth}{$l l l l l}
        \rowstyle{\normalfont\bfseries} #1 & \rowstyle{\normalfont\bfseries} #2 & \rowstyle{\normalfont\bfseries} #3 & \rowstyle{\normalfont\bfseries} #4 \rowstyle{\normalfont\bfseries} #5 \\
}{
    \endtabularx
}

\definecolor{codebox_bg}{HTML}{D9DCFF}
\definecolor{codebox_comment}{HTML}{004D25}
\newcommand{\codebox}[1]{\colorbox{codebox_bg}{\scriptsize\ttfamily\minibox{#1}}}

% include custom commands
% sets the default font to Times Roman
%\usepackage{mathptmx}

% sets sans-serif font to Hel­vetica
%\usepackage{helvet}

% set default font family
%\renewcommand{\familydefault}{\sfdefault}

% Ubuntu font
\usepackage[regular,light]{ubuntu} 



% contents
\begin{document}

\maketitle

\center
\vspace{2.5cm}
Source code is available at \href{https://github.com/Martchus/latex-summary}{https://github.com/Martchus/latex-summary}
\flushleft

\clearpage

\begingroup
\def\addvspace#1{}
\tableofcontents
\endgroup

\clearpage

\section{Document-wide commands}    
    \begin{cmdtab}
        \bs documentclass[options]\{class\} & sets document class (article/report/book/letter/scrartcl\dots) and related options (10pt\dots{}12pt, fleqn/leqno, titlepage/notitlepage, twocolumn/twoside, a4paper/a5paper, draft, landscape) \\
        \bs usepackage[options]\{pkgname\} & imports commands from the specified package \\
        \bs input\{file\} & includes the specified file \\
        \bs include\{file\} & includes the specified file (new page) \\
        \bs author\{val\}, \bs title\{val\}, \bs date\{val\} & sets metadata (used by \bs maketitle) \\
        \bs begin\{document\}, \bs end\{document\} & defines boundries of the actual text
    \end{cmdtab}
    
    \subsection{Important packages}
        \begin{cmdtab}
            \bs usepackage[lng]\{babel\} & provides \texttt{\bs selectlanguage\{lng\}}, \texttt{\bs foreignlanguage\{lng\}\{text\}} which affect commands like \texttt{\bs today} \\
            \minibox{\bs usepackage[latin1/utf8/\dots] \\ ~~\{inputenc\}} & sets the specified input encoding \\
            \bs usepackage[T1/\dots]\{fontenc\} & sets the specified font encoding \\
            \bs usepackage\{nameref\} & provides named references via \texttt{\bs nameref\{\}}, see \fullref{subsec:labels} \\
            \bs usepackage\{amsmath\} & provides \fullref{section:math} \\
            \bs usepackage\{caption\} & provides \texttt{\bs captionof\{ \}}, see \fullref{section:floating_environments} \\
            \bs usepackage\{xcolor\} & provides colors (see \fullref{subsec:notations_colors}) \\
            \bs usepackage\{hyperref\} & provides bookmarks, hyperlinks and PDF specific settings, see also \fullref{section:pdf_tweaks}. \\
            \bs usepackage\{geometry\} & allows to set page geometry (see \fullref{subsec:geometry}) \\
            \bs usepackage\{microtype\} & tunes line breaks
        \end{cmdtab}
    
    \subsection{Geometry}
        \label{subsec:geometry}
        With the \texttt{geometry} package page geometry can be set using \texttt{\bs geometry\{a4paper,left=45mm,\dots\}}.
        
        \begin{cmdtabx}{Option}{Effect}
            left, right, bottom, top & specify borders \\
            bindingoffset & specifies the binding offset \\
            includehead, includefoot & whether head and foot nodes include borders \\
        \end{cmdtabx}

\section{Escaping}
    \label{section:escaping}
    The following characters have special meaning and must be escaped: \framebox[1.1\width][c]{\texttt{\& \% \$ \# \_ \{ \} \textasciitilde \textasciicircum \textbackslash}}
    \begin{cmdtabxxxx}{Character(s)}{Escape with}
        \& \% \$ \# \_ \{ \} & \textnormal{backslash, eg. }\bs\& \\
        \textasciitilde & \bs textasciitilde \\
        \textasciicircum & \bs textasciicircum \\
        \textbackslash & \bs textbackslash
    \end{cmdtabxxxx}

    \paragraph{Notes}{
        \begin{itemize}
            \item Defining shortcuts might be useful, eg.: \texttt{\bs newcommand\{\bs bs\}\{\bs textbackslash\}}
            \item For including source code, see section \ref{section:including_source_code}. For special characters, see section \ref{subsec:special_characters}.
            \item The following characters mustn't be escaped: \framebox[1.1\width][c]{[ ] ( ) / ! ? * :}
            \item The following special characters can be used as label/color IDs (\textit{without} escaping): \framebox[1.3\width][c]{\_ :}
        \end{itemize}
    }

\section{Document structure}
    \begin{cmdtab}
        \bs part\{title\} & level -1 in book and report, level 0 in article \\
        \bs chapter\{title\} & level 0 in book and report \\
        \bs section\{title\} & level 1 \\
        \bs subsection\{title\} & level 2 \\
        \bs subsubsection\{title\} & level 3 \\
        \bs paragraph\{title\} & level 4 \\
        \bs subparagraph\{title\} & level 5 \\
        \bs appendix & starts the appendix \\
        \bs maketitle & generates the cover sheet \\
        \bs tableofcontents & generates the table of contents (see also \fullref{subsec:man_toc})
    \end{cmdtab}
    
    \subsection{Labels/anchors and references}
        \label{subsec:labels}
        \begin{cmdtab}
            \bs label\{anchor\_id\} & defines an anchor \\
            \bs ref\{anchor\_id\} & prints the index of the specified anchor \\
            \bs nameref\{anchor\_id\} & prints the name of the specified anchor \\
            \bs pageref\{anchor\_id\} & prints the page of the specified anchor \\
        \end{cmdtab}

        It might be useful to combine \texttt{\bs ref} and \texttt{\bs nameref}: \\
        \texttt{\bs newcommand\{\bs fullref\}[1]\{\bs ref\{\#1\}:\textasciitilde\bs nameref\{\#1\}\}}
    
    \subsection{Manipulate the table of contents}
        \label{subsec:man_toc}
        \begin{cmdtab}
            \bs section[toc\_title]\{title\} & sets a toc-specific title \\
            \bs section*\{title\} & disable index and appearance in toc \\
            \bs addcontentsline\{toc\}\{section\_level\}\{text\_entry\} & adds an additional entry
        \end{cmdtab}

    \subsection{Manual breaks}
        \begin{itemize}
            \item Manual line breaks can be inserted using \texttt{\bs\bs} or \texttt{\bs newline}. For a line break in a table cell a \texttt{\bs minibox} from
                  the \texttt{minibox} package can be used.
            \item A new paragraph is achieved by inserting an empty line or using the \texttt{\bs par} command.
            \item A page break can be inserted using \texttt{\bs newpage}.
            \item A page break can be inserted using \texttt{\bs clearpage} which also forces \LaTeX{} to print all remaining \fullref{section:floating_environments}.
        \end{itemize}
    
    \subsection{Example}
        \lstinputlisting[
            language={[LaTeX]tex},
            basicstyle={\ttfamily\footnotesize},
            keywordstyle={\color{blue}},
            backgroundcolor={\color{codebox_bg}},
            commentstyle={\color{codebox_comment}},
            numbers={left}
        ]{
            fragments/example_doc.tex
        }

\section{Spacing and ellipsis}

    \subsection{Spacing between lines}
        These commands requires the \texttt{setspace} package.

        \begin{cmdtab}
            \bs singlespacing & sets line spacing to 1.0 \\
            \bs onehalfspacing & sets line spacing to 1.5 \\
            \bs doublespacing & sets line spacing to 2.0 \\
            \minibox{\bs begin\{singlespace\}, \\ \bs begin\{onehalfspace\}, \dots, \\ \bs begin\{spacing\}\{factor\}} & begins environmenmt with specific line spacing \\
        \end{cmdtab}
    
    \subsection{Paragraphs}
        \begin{itemize}
            \item Spacing can be controlled with \texttt{\bs setlength\{\bs parskip\}\{spacing\}}.
            \item Indention can be controlled with \texttt{\bs setlength\{\bs parindent\}\{indent\}}.
        \end{itemize}

    \subsection{Miscellaneous}
        \begin{cmdtab}
            \bs dots & inserts ellipsis: \dots \\
            \bs hspace\{space\} & inserts horizontal space \\
            \bs vspace\{space\} & inserts vertical space \\
            \bs hfill \bs hrulefill \bs dotfill & horizontal filling \\
            \bs vfill & vertical filling \\
        \end{cmdtab}

\section{Alignment}
    \begin{cmdtabx}{Environment}{Alignment}
        flushleft & left \\
        flushright & right \\
        center & center \\
    \end{cmdtabx}

\section{Quotes and footnotes}
    \begin{cmdtab}
        \bs begin\{quote\} \dots \bs end\{quote\} & wraps a quote \\
        \bs footnote\{text\} & makes a footnote with the specified text
    \end{cmdtab}

\section{Floating environments}
    \label{section:floating_environments}
    
    \begin{cmdtabx}{Environment}{Use}
        table & to include, see \fullref{subsec:floating_tables} \\
        figure & to include graphics with \texttt{\bs includegraphics}, see \fullref{section:graphics} \\
        lstlisting & to include source code, see \fullref{section:including_source_code}
    \end{cmdtabx}

    \subsection{Preferred position}
        \label{subsec:floating_position}
        \begin{cmdtabx}{Option}{Effect}
            h & here \\
            t & top of page \\
            b & bottom of page \\
            p & separate page \\
            ! & increases the priotiry
        \end{cmdtabx}

        These options can be specified as usual in square brackets and might be combined.

    \subsection{Useful commands in floating environments}
        \begin{cmdtab}
            \bs centering & centers the environment \\
            \bs caption[toc\_text]\{text\} & inserts a description for the figure/table/\dots \\
            \bs label & see \fullref{subsec:labels}, must be \textbf{after} the \texttt{\bs caption} command \\
            \minibox{\bs captionof\{floating\_env\_type\} \\ ~ [toc\_text]\{text\}} & same as \texttt{\bs caption} but allows to specify the type of the floating environment, requires the \texttt{caption} package
        \end{cmdtab}

    \subsection{Parameter}
        \begin{cmdtabx}{Command}{Meaning}
            \bs topfraction & fraction for floats at the beginning of a page \\
            \bs bottomfraction & fraction for floats at the end of a page \\
            \bs textfraction & minimum fraction for text on a a page
        \end{cmdtabx}

        Parameter can be changed using eg. \texttt{\bs renewcommand\{\bs topfraction\}\{0.6\}}.

\section{Notations}

    \subsection{Units}
        \begin{itemize}
            %\setlength\itemsep{-1em}
            \item \texttt{mm, cm, in} \dots : milimeter, centimeter, inch
            \item \texttt{pt} : 0.3515 mm
            \item \texttt{pc} : 12 pt
            \item \texttt{ex}, \texttt{em} : height of small \texttt{x}, width of capital \texttt{M}
            \item \texttt{\bs baselineskipt} : height of a line
        \end{itemize}

    \subsection{Colors}
        \label{subsec:notations_colors}
        \begin{itemize}
            %\setlength\itemsep{-1em}
            \item require the package \texttt{xcolor}.
            \item pre-defined colors: \texttt{\textcolor{blue}{blue} \textcolor{violet}{violet} \textcolor{green}{green} \textcolor{red}{red} ...}
            \item custom colors: \texttt{\bs definecolor\{custom\_name\}\{scheme\}\{values\} }
                \begin{itemize}
                    %\vspace{-5mm}
                    %\setlength\itemsep{-1em}
                    \item schemes: \texttt{rgb cmyk HTML} \dots
                    \definecolor{red1}{HTML}{AA0000} \definecolor{blue1}{rgb}{0.1,0.1,1.0}
                    \item examples: \texttt{ \textcolor{red1}{\bs definecolor\{red1\}\{HTML\}\{AA0000\}} \\
                        \phantom{examples: } \textcolor{blue1}{\bs definecolor\{blue1\}\{rgb\}\{0.1,0.1,1.0\}} }
                \end{itemize}
            \item mixing colors: \texttt{ \textcolor{red!50}{red!50} \textcolor{blue!70!green!50}{blue!70!green!50} }
        \end{itemize}

\section{Mathematical stuff}
    \label{section:math}
    The \texttt{amsmath} package must be included for most commands and environments.

    \subsection{Environments}
        \begin{cmdtab}
            \$ some formula \$ & defines an inline formula \\
            \bs begin\{equation\} \dots{} \bs end\{equation\} & defines a \textit{single}-line formula block \\
            \bs begin\{align\} \dots{} \bs end\{align\} & defines a \textit{multi}-line formula block
        \end{cmdtab}

    \subsection{Mathematical notations}
        \label{subsec:math_notations}
        \begin{cmdtab}
            \bs frac\{numerator\}\{denominator\} & fraction: $ \frac{\mathrm{numerator}}{\mathrm{denominator}} $\\
            \bs sqrt[n]\{a\} & root: $ \sqrt[n]{a} $\\
            \bs cos(45\^\{\bs circ\}) = \bs cos(\bs pi/4) & $ \cos(45^{\circ}) = \cos(\pi/4) $ \\
            \^ \{up\} \_\{down\} & $ ^{up} _{down} $ \\
            \bs sum\_\{n=1\}\^\{12\}\{f(x\_n)\} & sum: $ \sum_{n=1}^{12}{f(x_n)} $ \\
            \bs prod \dots & product: $ \prod_{}^{} \dots $ \\
            \bs int \dots & integral: $ \int_{}^{} \dots $ \\
            \bs iint \dots & integral of integral: $ \iint_{}^{} \dots $ \\
            \bs lim \dots & limes: $ \lim \dots $ \\
            \bs left( \bs right) & scaled brackets (can also be used with \texttt{\{\} [] <>}) \\
            \bs bs begin\{cases\} \dots{} \bs end\{cases\} & defines cases, eg. $
                \begin{cases}
                        p & p \neq 0 \\
                        \infty & p = 0 \\
                \end{cases} $
        \end{cmdtab}

    \subsection{Letters and symbols}
        \label{subsec:math_symbols}
        \begin{cmdtab}
            \bs alpha \bs beta & greek letters: $ \alpha , \beta $ \\
            \bs neq \bs leq \bs geq & equation signs: $ \neq \leq \geq $ \\
            \bs in \bs notin & (not) in set: $ \in \notin $ \\
            \bs infty & infinity: $ \infty $ \\
            \bs phantom \{ \} & spacing in size of argument \\
            \bs mathrm\{ \} & roman font in forumal \\
            \minibox{\bs leftarrow \bs rightarrow \\ \bs longleftarrow \bs leftrightarrow} & arrows: $ \leftarrow \rightarrow \longleftarrow \leftrightarrow $ \\
            \bs overrightarrow\{x\} & $ \overrightarrow{x} $
        \end{cmdtab}

    \subsection{Spacing}
        \begin{cmdtabx}{Command}{Space}
            \bs, \bs thinspace & $ \rightarrow \thinspace \leftarrow $ \\
            \bs: \bs medspace & $ \rightarrow \medspace \leftarrow $ \\
            \bs; \bs thickspace & $ \rightarrow \thickspace \leftarrow $ \\
            \bs enskip & $ \rightarrow \enskip \leftarrow $ \\
            \bs quad & $ \rightarrow \quad \leftarrow $ \\
            \bs qquad & $ \rightarrow \qquad \leftarrow $ \\
            \bs! \bs negthinspace & $ \rightarrow \negthinspace \leftarrow $ \\
            \bs negmedspace & $ \rightarrow \negmedspace \leftarrow $ \\
            \bs negthickspace & $ \rightarrow \negthickspace \leftarrow $ \\
        \end{cmdtabx}

\section{Font}
    \subsection{Family and style}
    
        \begin{cmdtab}
            \bs rmfamily, \bs tesubsubsectionxtrm\{ \} & \textrm{roman} \\
            \bs sffamily, \bs textsf\{ \} & \textsf{sans-serif} \\
            \bs ttfamily, \bs texttt\{ \} & \texttt{typewriter} \\
            \bs scshape, \bs textsc\{ \} & \textsc{small-caps} \\
            \bs itshape, \bs textit\{ \} & \textit{italic} \\
            \bs bfseries, \bs textbf\{ \} & \textbf{bold} \\
            \bs fontfamily\{family\}\bs selectfont & specifies font family of following text
        \end{cmdtab}
        
        \subsubsection{Overriding defaults}
            \begin{cmdtabxx}{Variable}{Default}{Activated by}
                \bs familydefault & \bs rmdefault & \bs normalfont, \bs textnormal\{ \} \\
                \bs rmdefault & cmr & \bs rmfamily, \bs textrm\{ \} \\
                \bs ttdefault & cmtt & \bs ttfamily, \bs texttt\{ \} \\
                \bs scdefault & cmss & \bs sffamily, \bs textsf\{ \} \\
                \bs seriesdefault & m & \bs normalfont, \bs textnormal\{ \} \\
                \bs mddefault & m & \bs mdseries, \bs textmd\{ \} \\
                \bs bfdefault & bx & \bs bfseries, \bs textbf\{ \}
            \end{cmdtabxx}
            
            The listed variables might be overriden with \texttt{\bs renewcommand\{\bs variable\}\{\bs newvalue\}}, eg.:
            
            \begin{cmdtab}
                \bs renewcommand\{\bs rmdefault\}\{\bs pbk\} & sets the default roman font to \textit{Bookman} (pbk) \\
                \bs renewcommand\{\bs familydefault\}\{\bs sfdefault\} & sets the general default font to the default sans-serif font
            \end{cmdtab}
            
        \subsubsection{Setting fonts with packages}

            \begin{cmdtabxxxxx}{Package}{Text font}{Sans}{Typewriter}{Math}
                \textit{none} & CM Roman & CM SansSerif & CM Typewriter & CM Roman \\
                mathpazo & Palatino & & & $\approx$Palatino \\
                mathptmx & Times & & & $\approx$Times \\
                helvet & & Helvetica & & \\
                avant & & Avant Garde & & \\
                courier & &  & Courier & \\
                chancery & Zapf Chancery & & & \\
                bookman & Bookman & Avant Garde & Courier & \\
                newcent & New Century Schoolbook & Avant Garde & Courier & \\
                charter & Charter &  & &
            \end{cmdtabxxxxx}
        
        \subsubsection{Common font family names}
            \begin{cmdtabx}{Abbreviation}{Font name}
                cmr & CM Roman \\
                ppl & Palatino \\
                ptm & Times Roman \\
                pzc & Zapf Chancery \\
                pbk & Bookman \\
                phv & Helvetica
            \end{cmdtabx}

    \subsection{Size}
        \begin{cmdtab}
            \bs tiny & \tiny tiny font (5 pt) \\
            \bs scriptsize\ & \scriptsize very small font (7 pt) \\
            \bs footnotesize & \footnotesize quite small font (8 pt) \\
            \bs small & \small small font (9 pt) \\
            \bs normalsize & \normalsize normal font (10 pt) \\
            \bs large & large font (12 pt) \\
            \bs Large & larger font (14.4 pt) \\
            \bs LARGE & very large font (17.28 pt) \\
            \bs huge & huge font (20.74 pt) \\
            \bs Huge & largest font (24.88 pt) \\
        \end{cmdtab}
    
    \subsection{Color}
    
        \begin{cmdtab}
            \bs color\{color\} & sets the color \\
            \bs normalcolor & resets the color \\
            \bs textcolor\{color\}\{text\} & sets the color of the specified text \\
            \bs pagecolor\{color\} & sets the page background color \\
        \end{cmdtab}
        
    \subsection{Sections, paragraphs, \dots}
        These commands require the \texttt{sectsty} package.

        \begin{cmdtab}
            \bs allsectionsfont\{any\_latex\_cmd\} & allows to set font family/style for all sections \\
            \minibox{\bs subsectionfont\{any\_latex\_cmd\}, \\ \bs paragraphfont\{any\_latex\_cmd\}, \dots} & allows to set font family/style for subsections, paragraphs, \dots \\
        \end{cmdtab}

    \subsection{Special characters}
        \label{subsec:special_characters}

        \subsubsection{Quotation marks}
            \begin{cmdtab}
                
            \end{cmdtab}
        
        \subsubsection{Miscellaneous}
            \begin{itemize}
                \item with \texttt{textcomp} package: \texttt{\bs textcopyright} \textcopyright, \texttt{\bs texttrademark} \texttrademark, \texttt{\bs textcelsius} \textcelsius, \texttt{\bs texteuro}, \dots
                \item official {\euro}-symbol with \texttt{\bs usepackage[official,right]\}\{eurosym\}}: \texttt{\bs euro}, \texttt{\bs EUR\{123,45\}}
                \item with \texttt{pifont} package: \texttt{\bs ding\{sym\_num\}}, \texttt{\bs Pisymbol\{sym\_font\}\{sym\_num\}}
                \item using inline math environment: \texttt{\$ \bs rightarrow \bs leftrightarrow \$} $ \rightarrow \leftrightarrow $, see also section \ref{subsec:math_symbols}
                \item see also \fullref{section:escaping}
            \end{itemize}

\section{Enumerations}
    \label{section:enumerations}
    
    \begin{cmdtab}
        \bs begin\{itemize\} \dots \bs end\{itemize\} & defines enumeration block \textit{without} numbering \\
        \bs begin\{enumerations\} \dots \bs end\{enumerations\} & defines enumeration block \textit{with} numbering (use of \texttt{\bs label} is possible) \\
        \bs setlength\bs itemsep\{spacing\} & controls spacing between items \\
        \bs item[icon] & starts an item
    \end{cmdtab}

\section{Boxes}

    \begin{cmdtab}
        \bs fbox\{text\} & \fbox{\footnotesize simple framed box} \\
        \bs mbox\{text\} & \mbox{\footnotesize simple unframed box} \\
        \bs framebox[width][alignment]\{text\} & \framebox[1.5\width][c]{\minibox{\footnotesize framed box with alignment \\ \footnotesize and width (here "\texttt{1.5\bs width}")}}  \\
        \bs makebox[width][alignment]\{text\} & \makebox[1.5\width][c]{\minibox{\footnotesize unframed box with alignment \\ \footnotesize and width (here "\texttt{1.5\bs width}")}} \\
        \bs colorbox\{bg\_color\}\{text\} & \colorbox{yellow}{\footnotesize unframed box with background color} \\
        \bs colorbox\{frame\_color\}\{bg\_color\}\{text\} & \fcolorbox{blue}{white!50!green!50}{\footnotesize framed box with colored background and frame} \\
        \bs minibox\{text\} & unframed box which allows manual line breaks (requires \texttt{minibox} package) \\
        \bs rule[vertical\_offset]\{width\}\{height\} & \rule{2.5cm}{0.25cm} (invisible) padding (when either \texttt{width} or \texttt{height} is zero)  \\
    \end{cmdtab}
    
    \subsection{fancybox}
        These commands require the \texttt{fancybox} package.

        \begin{cmdtab}
            \bs ovalbox\{text\} & \ovalbox{\footnotesize oval box} \\
            \bs Ovalbox\{text\} & \Ovalbox{\footnotesize oval box (thick border)} \\
            \bs shadowbox\{text\} & \shadowbox{\footnotesize shadowed box} \\
            \bs doublebox\{text\} & \doublebox{\footnotesize box with double border} \\
            \bs cornersize\{corner\_size\} & manipulates the corner size \\
            \bs raisebox\{raise\}\{text\} & allows to \raisebox{0.5ex}{raise} and to \raisebox{-0.5ex}{lower} (negative raise) \\
            \bs underline\{text\} & \underline{underline} \\
            \bs underbar\{text\} & \underbar{underbar} \underline{\underbar{underline and underbar}} \\
        \end{cmdtab}
    
    \subsection{Minipages}
        Allow side by side positioning; can be used withing \texttt{float} environment; require the \texttt{minipage} package.
        
        \begin{cmdtab}
            \bs begin\{minipage\}[vert\_align]\{width\} \dots \bs end\{minipage\} & defines minipage \\
        \end{cmdtab}

        \paragraph{Vertical alignment}{
            either t, c or b for top, center or bottom
        }
        subsubsection
        \paragraph{Width}{
            use eg. \texttt{0.5 \bs textwidth} for 50 \% of the available with
        }

\section{Graphics}
    \label{section:graphics}
    *.pdf, *.jpg, *.png files can be included with \texttt{\bs includegraphics[options]\{image\}}.

    \begin{cmdtabx}{Option}{Note}
        viewport=x1 y2 x2 y2, clip & cropping (in $ \frac{1}{72} $ inch, relative to bottom left corner) \\
    \end{cmdtabx}
    
    \subsection{Annotations}
        Graphics can be annotated using the \texttt{overpic} environment of the \texttt{overpic} package.
        
        \begin{lstlisting}[
            caption={Annotation example},
            language={[LaTeX]tex},
            basicstyle={\ttfamily\footnotesize},
            keywordstyle={\color{blue}},
            backgroundcolor={\color{codebox_bg}},
            numbers={left}
            ]
\begin{overpic}[tics=10,height=7cm]{robot}
    \put(80 ,32){detector}
    \put(87 ,36){\vector(-1,4){1.5}}
    \put(50 ,18){source}
    \put(77 ,19){\vector(1,0){7}}
    \put(15 ,50){valve}
    \put(30 ,50){\vector(1,-1){5}}
    \put(19 ,13){{\small A1}}
\end{overpic}
        \end{lstlisting}

\section{Tables}
    For basic tables no packages need to be included. However the packages \texttt{tabularx}, \texttt{array} and \texttt{longtable} provide extended features.

    \begin{cmdtab}
        \minibox{ \bs begin\{tabular\}[position]\{column\_declarations\} \\ \dots \bs end\{tabular\} } & defines a table \\
        \minibox{ \bs begin\{tabular*\}\{width\}[position]\{column\_declarations\} \\ \dots \bs end\{tabular*\} } & defines a table with the specified with \\
        \minibox{ \bs begin\{array\}[position]\{column\_declarations\} \\ \dots \bs end\{array*\} } & defines a table (for formulas) \\
        \minibox{ \bs begin\{tabbing\} \\ \dots \bs end\{tabbing\} } & defines a tabulator-based table \\
        \minibox{ \bs begin\{longtable\}[position]\{column\_declarations\} \\ \dots \bs end\{longtable\} } & defines a table which is able to expand multiple pages (can not be used in a floating environment) \\
        \bs listoftables & prints the list of tables
    \end{cmdtab}
    
    \subsection{Column definitions}
        \begin{cmdtabx}{Syntax}{Description}
            l, r, c & specifies the alignment \\
            p\{with\} & specifies with \\
            | & defines vertical line between columns \\
            @\{col\_spacing\} & sets the column spacing \\
            *\{count\}\{definition\} & inserts \texttt{definition} \texttt{count} times
        \end{cmdtabx}
    
    \begin{lstlisting}[
            caption={Example table},
            language={[LaTeX]tex},
            basicstyle={\ttfamily\footnotesize},
            keywordstyle={\color{blue}},
            backgroundcolor={\color{codebox_bg}},
            numbers={left},
            float=h
        ]
\begin{tabular}{|lcr|}
    \hline
    Color & Shape & Number \\
    red & rect & 100 \\
    blue & circle & 99 \\
    \hline
\end{tabular}
    \end{lstlisting}

    \begin{lstlisting}[
            caption={Example tabulator-based table},
            language={[LaTeX]tex},
            basicstyle={\ttfamily\footnotesize},
            keywordstyle={\color{blue}},
            backgroundcolor={\color{codebox_bg}},
            numbers={left},
            float=h
        ]
\begin{tabbing}
    Distributions: \ \= Column 1 \= Column 2 XX\kill \\
    Distributions:   \> Name \= Packaging system \\
                     \> Arch Linux \= pacman \\
                     \> Debian \= APT
\end{tabbing}
    \end{lstlisting}
    
    \subsection{Floating tables}
        \label{subsec:floating_tables}
        \begin{cmdtab}
            \minibox{ \bs begin\{table\} \dots \bs end\{table\} } & defines a floating environment for embedding the actual table \\
            \bs centering & sets alignment of the \texttt{table} environment to center \\
            \bs caption & specifies the caption of the floating table \\
            \bs label & defines an anchor referring to the floating table
        \end{cmdtab}
        
        \paragraph{Remarks}{
            The \texttt{table} environment defines only the floating element. The actual table is still defined using the \texttt{tabular} environment.
        }

\section{Including source code}
    \label{section:including_source_code}
    The \texttt{lstlisting} environment and assoiciated commands require the \texttt{listing\textbf{s}} package.

    \begin{cmdtab}
        \bs begin\{verbatim\} \dots \bs end\{verbatim\} & ignores \LaTeX{} syntax, sets typewriter font \\
        \bs begin\{lstlisting\}[options] \dots \bs end\{lstlisting\} & advanced version of verbatim, see \fullref{subsec:listing_options} \\
        \bs lstinputlisting[options]\{file\} & same as \texttt{lstlisting} environment, but allows including an external file
    \end{cmdtab}
    
    \subsection{lstlisting options}
        \label{subsec:listing_options}
        \begin{cmdtabxy}{Option}{Example/values}{Description}
            caption= & \{Some C++ code\} & caption (with number) \\
            label= & \{anchor\_name\} & defines an anchor, see \fullref{subsec:labels} \\
            title= & \{More C++ code\} & caption (without number) \\
            language= & \{[Visual]C++\}/\{[LaTeX]tex\} & programming language \\
            breaklines= & true/false & enables/disables line breaks \\
            basicstyle= & \{\bs ttfamily\bs footnotesize\} & defines the basic style \\
            keywordstyle= & \{\bs color\{blue\}\} & sets the style of keywords \\
            commentstyle= & \{\bs color\{green\}\} & sets the style of comments \\
            stringstyle= & \{\bs color\{brown\}\} & sets the style of string literal \\
            backgroundcolor= & \{\bs color\{yellow\}\} & sets the background color \\
            frame= & \minibox{ none/leftline/topline/ \\ bottomline/lines/shadowbox } & specifies the appearance of the frame \\
            numbers= & left/right & enables line numbers \\
            inputencoding= & latin1 & specifies the input encoding \\
            float= & \textnormal{see \fullref{subsec:floating_position}} & enables floating
        \end{cmdtabxy}

\section{Biblography}
    \subsection{Manual bibliography}
    
        \begin{cmdtab}
            \minibox{\bs begin\{thebibiliography\}\{abbr\_length\} \\ \dots \bs end\{thebibiliography\}} & defines a manual bibliography \\
            \bs bibitem\{key\} title of book, author \dots & starts an item
        \end{cmdtab}

    \subsection{Automatically generated bibliography}
    
        \begin{cmdtab}
            \bs biblographystyle\{style\_file\} & sets the bibliography layout (see \fullref{subsec:bibtex_styles}) \\
            \bs bibliography\{bib\_file1, bib\_file2, \dots\} & makes the bibliography \\
            \bs cite[text]\{key\} & reference to biblography entry with \textit{text} \\
            \bs nocite\{key1, key2\} & ensures the specified entries occur in the biblography without producing a reference \\
            \bs nocite\{*\} & ensures all entries occur in the biblography without producing any references
        \end{cmdtab}

        \subsubsection{Styles}
            \label{subsec:bibtex_styles}
        
            \begin{cmdtabx}{Name}{Note}
                plain & alphabetical order, numeric marks \\
                unsrt & sorted by the occurrence of references, numeric marks \\
                alpha & alphabetical order, marks with author and year  \\
                natdin & alphabetical order, marks with full author name and year according DIN 1505 part 2 (required \texttt{natbib} package)
            \end{cmdtabx}    

        \subsubsection{*.bib-File example}
            \begin{minipage}{0.5\textwidth}
                \codebox{
                    @book\{ entry\_id, \\
                        \phantom{  } author = \{Goossens, Michel and Mittelbach, Frank\}, \\
                        \phantom{  } title = \{Der LaTeX-Begleiter\}, \\
                        \phantom{  } publisher = \{Pearson Studium\}, \\
                        \phantom{  } address = \{M\{\bs "\{u\}\}nchen\}, \\
                        \phantom{  } year = \{2005\}, \\
                        \phantom{  } \\
                        \phantom{  } \dots \\
                    \}
                }
            \end{minipage}%
            \begin{minipage}{0.5\textwidth}
                \codebox{
                    @article\{ entry\_id, \\
                        \phantom{  } author = \{Neubauer, Marion\}, \\
                        \phantom{  } title = \{Mikrotypographie-\{R\}egeln, \{T\}eil 1\}, \\
                        \phantom{  } journal = \{Die \{T\}eXnische \{K\}om\{\bs "\{o\}\}die\}, \\
                        \phantom{  } number = \{4\}, \\
                        \phantom{  } pages = \{23--40\}, \\
                        \phantom{  } year = \{1996\}, \\
                        \phantom{  } \dots \\
                    \}
                }
            \end{minipage}
            
            \paragraph{Other entry classes}{
                \texttt{@booklet}, \texttt{@conference}, \texttt{@manual}, \texttt{@masterthesis}, \texttt{@misc}, \\
                \texttt{@string\{abbreviation\_id = "Text"\}}, ~\dots
            }
    
        \subsubsection{Compilation steps}
            \begin{enumerate}
                %\setlength\itemsep{-1em}
                \item \texttt{pdflatex}: generates *.aux-file (for \texttt{\bs cite}-commands)
                \item \texttt{bibtex}: generates *.bbl-file (from *.aux- and *.bib-file)
                \item \texttt{pdflatex}: can now generate bibliography (from *.bbl-file)
                \item \texttt{pdflatex}: can now generate references to bibliography
            \end{enumerate}

\section{Index}
    These commands require the \texttt{makeidx} package and \texttt{\bs makeindex} in the header.

    \begin{cmdtab}
        \bs printindex & prints the index \\
        \bs index\{index\_entry\} & defines an index entry, see \fullref{subsec:index_syntax} \\
    \end{cmdtab}
    
    \subsection{Syntax}
        \label{subsec:index_syntax}
        
        \begin{cmdtabx}{Symbol}{Effect}
            @ & divides \textit{key} and \textit{entry}, eg. \texttt{\bs index\{key@entry\}} (\textit{entry} is sorted by \textit{key}) \\
            ! & divides main entry and secondary entry, eg. \texttt{\bs index\{main\_entry!sec\_entry\}} \\
            | & starts command which is applied to page number, eg. \texttt{\bs index\{important|textbf\}} \\
            |(\dots|) & starts/ends page range, eg. \texttt{\bs index\{entry|(\}}\dots\texttt{\bs index\{entry|)\}} $\rightarrow$ key, 7-9 \\
            " & escape character, eg. \texttt{\bs index\{"@\}} $\rightarrow$ @, 5
        \end{cmdtabx}
    
    \subsection{Compilation steps}
        \begin{enumerate}
            %\setlength\itemsep{-1em}
            \item \texttt{pdflatex}: generates *.idx-file (for \texttt{\bs index}-commands)
            \item \texttt{makeindex}: generates *.ind-file (from *.idx- and *.ist-file)
            \item \texttt{pdflatex}: can now generate index
        \end{enumerate}

\section{Nomenclature/symbol table}
    These commands require the \texttt{nomencl} package and \texttt{\bs makenomenclature} in the header.
    
    \begin{cmdtab}
        \bs printnomenclature & prints the nomenclature \\
        \bs nomenclature\{symbol\}\{description\} & defines a symbol \\
    \end{cmdtab}
    
    \subsection{Compilation steps}
        \begin{enumerate}
            %\setlength\itemsep{-1em}
            \item \texttt{pdflatex}: reads \texttt{\bs nomenclature}-commands
            \item \texttt{nomencl}: generates *.nls and *.ilg files
            \item \texttt{pdflatex}: can print nomenclature
        \end{enumerate}

\section{Customization}
    \subsection{Commands}
        \begin{cmdtab}
            \bs newcommand\{\bs cmd\_name\}[arg\_count]\{cmd\_content\} & defines a new command \\
            \bs renewcommand\{\bs cmd\_name\}[arg\_count]\{cmd\_content\} & redefines an existing command \\
        \end{cmdtab}
    
    \subsection{Environments}
        \begin{cmdtab}
            \bs newenvironment\{\bs env\_name\}\{begin\_env\_code\}\{end\_env\_code\} & defines a new environment \\
        \end{cmdtab}
    
    \subsection{Counter}
        \begin{cmdtab}
            \bs newcounter\{counter\_name\} & defines a new counter which is initialized with 0 \\
            \bs arabic\{counter\_name\} & prints the counter value with Arabic digits \\
            \bs roman\{counter\_name\} & prints the counter value with Roman digits \\
            \bs value\{counter\_name\} & returns the counter value \\
            \bs setcounter\{counter\_name,value\} & assigns the counter to the specified value \\
            \bs addtocounter\{counter\_name,value\} & increments the counter by the specified value \\
            \bs stepcounter\{counter\_name\} & increments the counter by one \\
        \end{cmdtab}
    
    \subsection{Lengths}
        \begin{cmdtab}
            \bs setlength\{\bs length\_name\}\{length\} & defines a new length \\
            \bs addtolength\{\bs length\_name\}\{length\_increment\} & adds \\
            \bs settowidth\{\bs length\_name\}\{some\_text\} & defines a new length with the length of the specified text \\
        \end{cmdtab}    
    
    \subsection{Comparsion operations}
        Requires the \texttt{ifthen} package.

        \paragraph{Usage}{
            \texttt{\bs ifthenelse\{condition\}\{"true" branch\}\{"false" branch\}}, \\
            eg. \texttt{\bs ifthenelse\{\bs value\{c1\} > \bs value\{c2\}\}\{\$c1 > c2\$\}\{\$c1 \bs le c2\$\}}
        }
            
        \paragraph{Loops}{
            \texttt{\bs whiledo\{condition\}\{code\}}
        }
    
    \subsection{Document classes}
        Create *.cls file, eg. \\
        \codebox{
            \bs ProvidesClass\{myclass\}[desc] \\
            \bs LoadClassWithOptions[a4paper,ngerman,twoside]\{article\} \\
            \bs RequiredPackage[ansinew]\{inputenc\} \\
            \bs RequiredPackage[T1]\{fontenc\} \\
            \bs RequiredPackage[ngerman]\{babel\} \\
            \bs RequiredPackage\{xcolor,graphics\} \\
        }
    
    \subsection{Packages}
        Create *.sty file, eg. \\
        \codebox{
            \bs ProvidesPackage\{mypackage\}[desc] \\
            \bs newenvironment\{\dots\}\{\dots\}\{\dots\} \\
            \bs newcommand\{\dots\}[\dots]\{\dots\} \\
            \dots
        }

\section{Koma-Script}
    Bundles misc classes and packages for European layout.
    
    \subsection{Classes}
        \begin{cmdtabxxxx}{Default class}{Koma-Script class}
            article & scrartcl \\
            report & scrreprt \\
            book & scrbook \\
            letter & scrlettr
        \end{cmdtabxxxx}

\section{PDF tweaks}
    \label{section:pdf_tweaks}

    The following commands are PDF specific and require the \texttt{hyperref} package
    which should be loaded as last package.

    \subsection{PDF specific configuration (example)}

        \begin{lstlisting}[
            language={[LaTeX]tex},
            basicstyle={\ttfamily\footnotesize},
            keywordstyle={\color{blue}},
            backgroundcolor={\color{codebox_bg}},
            numbers={left}
            ]
\hypersetup{%
    pdfauthor={The author},
    pdftitle={The title},
    pdfsubject={The subject},
    pdfkeywords={keyword1, keyword2, ...},
    pdfstartview={FitV},
    pdfview={FitH},
    pdfpagemode={FullScreen},
    colorlinks={true/false},
    urlcolor={some\_color},
    backref={true/false}
}
        \end{lstlisting}

    \subsection{Links and bookmarks}
        \begin{cmdtab}
            \bs href\{url\}\{text\} & makes a \href{https://github.com/Martchus}{link} with the specified text and url \\
            \bs url\{url\} & make a link with the specified url which is also used as link text \\
            \bs pdfbookmark[level]\{text\}\{anchor\_id\} & inserts a PDF bookmark
        \end{cmdtab}
        
        \paragraph{Note}{
            \nameref{section:escaping} of \# and \textasciitilde{} is not necessary.
        }
    
    \subsection{PDF inclusion}
        Can be done with the \texttt{\bs includepdf[options]\{document\_name\}} command which requires the \texttt{pdfpages} package.
        
        \begin{cmdtabx}{Option}{Specifies}
            pages & the pages to be included, eg. \texttt{pages=\{2-4; 10\}} \\
            nup & the number of (included) pages on one page, eg. \texttt{nup=\textit{x}{}x{}\textit{y}} \\
            landscape & whether landscape layout is used (\texttt{true} or \texttt{false})
        \end{cmdtabx}

\section{Presentations}
    \begin{itemize}
        \item document class: \texttt{beamer}
        \item each page is embedded in \texttt{\bs frame\{content\}} or \texttt{frame} environment
        \item presentation structure is defined using \texttt{\bs section[long\_heading]\{TOC\_heading\}, \bs subsection[]\{\}}\dots
    \end{itemize}
    
    \begin{cmdtab}
        \bs frame & wraps a page \\
        \bs frametitle\{title\} & sets the frame title \\
        \bs titlepage & makes the title page
    \end{cmdtab}

\section{Further information}
    Full \LaTeX{} documentation is available at the \href{https://www.ctan.org}{CTAN (Comprehensive TeX Archive Network) website}.

\end{document}
